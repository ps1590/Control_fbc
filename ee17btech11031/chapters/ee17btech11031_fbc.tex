\begin{enumerate}[label=\thesection.\arabic*.,ref=\thesection.\theenumi]
\numberwithin{equation}{enumi}

\item The CE BJT amplifier in Fig. \ref{fig:Original ckt} employs shunt–shunt feedback: Feedback resistor $R_{F}$ senses the output Voltage $V_{o}$ and provides a feedback current to the base node.
\brak{R_{f} = 56 k\ohm, R_{C} = 5.6 k\ohm, R_{S} = 10 k\ohm}

 \begin{figure}[!ht]
 	\begin{center}
 			\resizebox{\columnwidth}{!}{\begin{circuitikz}[american]
\ctikzset{tripoles/mos style/arrows}

\draw (0,0) node [ground]{};
\draw (0,0) to[vsource, l = $V{s}$] (0,3);
\draw (0,3) to[R = $R_{S}$]  (2,3) -- (5,3) ;
\draw (5.5,3) node[npn](npn1){};
\draw (npn1.E) -- (5.5,0) to node[ground]{} (5.5,0);
\draw (npn1.C)  to[R = $R_{C}$] (5.5, 7) ;
\draw (5.5,7) to[short, -o] (5.5,7);
\draw (2.5, 3) -- (2.5, 4) to[R = $R_{f}$] (5.5,4) to[short, -o] (6.5, 4);
\draw node at (7,4){$V_{o}$} ;
\draw node at(5.5,7.4) {$+15V$};
\draw node at(1.5,2){$I_{S}$};
\draw node at (1.8,4){$I_{f}$};
\draw node at (3.5,2){$I_{B}$};

\draw (npn1.C)++(2.75,-1.5) node[label={right:$R_{out}$}]{}--++(0,1)--++(-1.5,0)[->];
\draw (npn1.B)++(-2.5,-3) node[label={left:$R_{in}$}]{}--++(0,1.5)--++(0.5,0)[->];
\draw [-latex] (2,2.5) -- (.8,2.5);
\draw [-latex] (2.2,4.5) -- (2.2,3.5);
\draw [-latex] (3,2.5) -- (4,2.5);
\end{circuitikz}

}
 	\end{center}
 \caption{}
 \label{fig:Original ckt}
 \end{figure}
 
\item Draw the equivalent control system for fig. \ref{fig:Original ckt} 
\solution

 \begin{figure}[!ht]
 	\begin{center}
 			\resizebox{\columnwidth}{!}{\tikzstyle{input} = [coordinate]
\tikzstyle{output} = [coordinate]
\tikzstyle{block} = [draw, rectangle]
\tikzstyle{sum} = [draw, circle]

\begin{tikzpicture}[auto, node distance=2cm,>=latex']
    \node [input, name=input] {$I_s$};
    \node [sum, right of=input] (sum) {};
    \node [block, right of=sum] (controller) {$G$};
    \node [output, right of=controller] (output) {};
    \node [block, below of=controller] (feedback) {$H$};
    \draw [draw,->] (input) -- node {$I_{s}$} (sum);
    \draw [->] (sum) -- node {$I_{i}$} (controller);
    \draw [->] (controller) -- node [name=y] {$V_o$}(output);
    \draw [->] (y) |- (feedback);
    \draw [->] (feedback) -| node[pos=0.95]{$-$}  node [near end] {$I_f$} (sum);
    \draw [->] (feedback) -| node[pos=1.15]{$+$}  node [near end] {} (sum);
\end{tikzpicture}}
 	\end{center}
 \caption{}
 \label{fig: cs blk}
 \end{figure}
 
\item If $V_{s}$ has a zero dc component, find the dc collector current of the BJT. Assume the transistor H = 100.

 \solution 
 Since, $V_{E}$ = 0 and $V_{S}$ = $V_{BE}$
 \begin{align}
    I_{S} &= \frac{V_{BE}}{R_{S}}
     = \frac{0.7}{10 * 10^3}\\
    \implies I_{S} &= 0.07 mA
 \end{align}
 
 Applying KCL at feedback resistor output
\begin{multline}
    -V_{o} + V_{BE} + I_{f}R_{f} = 0\\
    \text{\brak{Since, I_{f} = I_{B} + I_{S}}}\\
    V_{o} = V_{BE} + \brak{I_{B} + I_{S}}R_{f}  \\
     = 0.7 + \brak{I_{B} + 0.07*10^{-3}}\brak{56*10^3}\\
    \implies V_{o}= \brak{56*10^3}I_{B} + 4.62
     \label{eq:ee17btech11031_fed1}
\end{multline}

 Applying KCL at collector node
 \begin{multline}
    \frac{V_{o} - 15}{5.6*10^3} + I_{C} + I_{f} = 0\\
     \text{\brak{Since,I_{C} = H I_{B}}}\\
     \frac{V_{o} - 15}{5.6*10^3} + H I_{B} + \brak{I_{B} + I_{S}} = 0\\
     \frac{V_{o} - 15}{5.6*10^3} + \brak{100 + 1}I_{B} + \brak{0.07*10^{-3}} = 0\\
     \implies V_{o} = 14.608 - \brak{565.5*10^3}I_{B} 
     \label{eq:ee17btech11031_fed2}
 \end{multline}
 Subtracting \ref{eq:ee17btech11031_fed1} from \ref{eq:ee17btech11031_fed2}, we get,
 \begin{align}
     I_{B} &= 16.06  \mu A\\
     I_{C} &= I_{E} = H I_{B}\\
     \text{Dc collector Current},  
     I_{C} &= 1.606  mA
 \end{align}
 
 \begin{table}[!ht]
\centering
\def\ifundefined#1{\expandafter\ifx\csname#1\endcsname\relax}


%%  Check for the \def token for inputed files. If it is not        %%
%%  defined, the file will be processed as a standalone and the     %%
%%  preamble will be used.                                          %%
\ifundefined{inputGnumericTable}

%%  We must be able to close or not the document at the end.        %%
	\def\gnumericTableEnd{\end{document}}


%%%%%%%%%%%%%%%%%%%%%%%%%%%%%%%%%%%%%%%%%%%%%%%%%%%%%%%%%%%%%%%%%%%%%%
%%                                                                  %%
%%  This is the PREAMBLE. Change these values to get the right      %%
%%  paper size and other niceties.                                  %%
%%                                                                  %%
%%%%%%%%%%%%%%%%%%%%%%%%%%%%%%%%%%%%%%%%%%%%%%%%%%%%%%%%%%%%%%%%%%%%%%

	\documentclass[12pt%
			  %,landscape%
                    ]{report}
       \usepackage[latin1]{inputenc}
       \usepackage{fullpage}
       \usepackage{color}
       \usepackage{array}
       \usepackage{longtable}
       \usepackage{calc}
       \usepackage{multirow}
       \usepackage{hhline}
       \usepackage{ifthen}


%%  End of the preamble for the standalone. The next section is for %%
%%  documents which are included into other LaTeX2e files.          %%
\else

%%  We are not a stand alone document. For a regular table, we will %%
%%  have no preamble and only define the closing to mean nothing.   %%
    \def\gnumericTableEnd{}

%%  If we want landscape mode in an embedded document, comment out  %%
%%  the line above and uncomment the two below. The table will      %%
%%  begin on a new page and run in landscape mode.                  %%
%       \def\gnumericTableEnd{\end{landscape}}
%       \begin{landscape}


%%  End of the else clause for this file being \input.              %%
\fi

%%%%%%%%%%%%%%%%%%%%%%%%%%%%%%%%%%%%%%%%%%%%%%%%%%%%%%%%%%%%%%%%%%%%%%
%%                                                                  %%
%%  The rest is the gnumeric table, except for the closing          %%
%%  statement. Changes below will alter the table's appearance.     %%
%%                                                                  %%
%%%%%%%%%%%%%%%%%%%%%%%%%%%%%%%%%%%%%%%%%%%%%%%%%%%%%%%%%%%%%%%%%%%%%%

\providecommand{\gnumericmathit}[1]{#1} 
%%  Uncomment the next line if you would like your numbers to be in %%
%%  italics if they are italizised in the gnumeric table.           %%
%\renewcommand{\gnumericmathit}[1]{\mathit{#1}}
\providecommand{\gnumericPB}[1]%
{\let\gnumericTemp=\\#1\let\\=\gnumericTemp\hspace{0pt}}
 \ifundefined{gnumericTableWidthDefined}
        \newlength{\gnumericTableWidth}
        \newlength{\gnumericTableWidthComplete}
        \newlength{\gnumericMultiRowLength}
        \global\def\gnumericTableWidthDefined{}
 \fi
%% The following setting protects this code from babel shorthands.  %%
 \ifthenelse{\isundefined{\languageshorthands}}{}{\languageshorthands{english}}
%%  The default table format retains the relative column widths of  %%
%%  gnumeric. They can easily be changed to c, r or l. In that case %%
%%  you may want to comment out the next line and uncomment the one %%
%%  thereafter                                                      %%
\providecommand\gnumbox{\makebox[0pt]}
%%\providecommand\gnumbox[1][]{\makebox}

%% to adjust positions in multirow situations                       %%
\setlength{\bigstrutjot}{\jot}
\setlength{\extrarowheight}{\doublerulesep}

%%  The \setlongtables command keeps column widths the same across  %%
%%  pages. Simply comment out next line for varying column widths.  %%
\setlongtables

\setlength\gnumericTableWidth{%
	53pt+%
	163pt+%
0pt}
\def\gumericNumCols{2}
\setlength\gnumericTableWidthComplete{\gnumericTableWidth+%
         \tabcolsep*\gumericNumCols*2+\arrayrulewidth*\gumericNumCols}
\ifthenelse{\lengthtest{\gnumericTableWidthComplete > \linewidth}}%
         {\def\gnumericScale{\ratio{\linewidth-%
                        \tabcolsep*\gumericNumCols*2-%
                        \arrayrulewidth*\gumericNumCols}%
{\gnumericTableWidth}}}%
{\def\gnumericScale{1}}

%%%%%%%%%%%%%%%%%%%%%%%%%%%%%%%%%%%%%%%%%%%%%%%%%%%%%%%%%%%%%%%%%%%%%%
%%                                                                  %%
%% The following are the widths of the various columns. We are      %%
%% defining them here because then they are easier to change.       %%
%% Depending on the cell formats we may use them more than once.    %%
%%                                                                  %%
%%%%%%%%%%%%%%%%%%%%%%%%%%%%%%%%%%%%%%%%%%%%%%%%%%%%%%%%%%%%%%%%%%%%%%

\ifthenelse{\isundefined{\gnumericColA}}{\newlength{\gnumericColA}}{}\settowidth{\gnumericColA}{\begin{tabular}{@{}p{53pt*\gnumericScale}@{}}x\end{tabular}}
\ifthenelse{\isundefined{\gnumericColB}}{\newlength{\gnumericColB}}{}\settowidth{\gnumericColB}{\begin{tabular}{@{}p{163pt*\gnumericScale}@{}}x\end{tabular}}

\begin{tabular}[c]{%
	b{\gnumericColA}%
	b{\gnumericColB}%
	}

%%%%%%%%%%%%%%%%%%%%%%%%%%%%%%%%%%%%%%%%%%%%%%%%%%%%%%%%%%%%%%%%%%%%%%
%%  The longtable options. (Caption, headers... see Goosens, p.124) %%
%	\caption{The Table Caption.}             \\	%
% \hline	% Across the top of the table.
%%  The rest of these options are table rows which are placed on    %%
%%  the first, last or every page. Use \multicolumn if you want.    %%

%%  Header for the first page.                                      %%
%	\multicolumn{2}{c}{The First Header} \\ \hline 
%	\multicolumn{1}{c}{colTag}	%Column 1
%	&\multicolumn{1}{c}{colTag}	\\ \hline %Last column
%	\endfirsthead

%%  The running header definition.                                  %%
%	\hline
%	\multicolumn{2}{l}{\ldots\small\slshape continued} \\ \hline
%	\multicolumn{1}{c}{colTag}	%Column 1
%	&\multicolumn{1}{c}{colTag}	\\ \hline %Last column
%	\endhead

%%  The running footer definition.                                  %%
%	\hline
%	\multicolumn{2}{r}{\small\slshape continued\ldots} \\
%	\endfoot

%%  The ending footer definition.                                   %%
%	\multicolumn{2}{c}{That's all folks} \\ \hline 
%	\endlastfoot
%%%%%%%%%%%%%%%%%%%%%%%%%%%%%%%%%%%%%%%%%%%%%%%%%%%%%%%%%%%%%%%%%%%%%%

\hhline{|-|-}
	 \multicolumn{1}{|p{\gnumericColA}|}%
	{\gnumericPB{\centering}\gnumbox{\textbf{Parameter}}}
	&\multicolumn{1}{p{\gnumericColB}|}%
	{\gnumericPB{\centering}\gnumbox{\textbf{Description}}}
\\
\hhline{|--|}
	 \multicolumn{1}{|p{\gnumericColA}|}%
	{\gnumericPB{\raggedright}\gnumbox[l]{$R_{in}$}}
	&\multicolumn{1}{p{\gnumericColB}|}%
	{\gnumericPB{\raggedright}\gnumbox[l]{Total Input Resistance}}
\\
\hhline{|--|}
	 \multicolumn{1}{|p{\gnumericColA}|}%
	{\gnumericPB{\raggedright}\gnumbox[l]{$R_{out}$}}
	&\multicolumn{1}{p{\gnumericColB}|}%
	{\gnumericPB{\raggedright}\gnumbox[l]{Total Output Resistance}}
\\
\hhline{|--|}
	 \multicolumn{1}{|p{\gnumericColA}|}%
	{\gnumericPB{\raggedright}\gnumbox[l]{$r_{\pi}$}}
	&\multicolumn{1}{p{\gnumericColB}|}%
	{\gnumericPB{\raggedright}\gnumbox[l]{Output resistance of NPN}}
\\
\hhline{|--|}
	 \multicolumn{1}{|p{\gnumericColA}|}%
	{\gnumericPB{\raggedright}\gnumbox[l]{$R_{f}$}}
	&\multicolumn{1}{p{\gnumericColB}|}%
	{\gnumericPB{\raggedright}\gnumbox[l]{Feedback resistance}}
\\
\hhline{|--|}
	 \multicolumn{1}{|p{\gnumericColA}|}%
	{\gnumericPB{\raggedright}\gnumbox[l]{$R_{i}$}}
	&\multicolumn{1}{p{\gnumericColB}|}%
	{\gnumericPB{\raggedright}\gnumbox[l]{Input resistance of G circuit}}
\\
\hhline{|--|}
	 \multicolumn{1}{|p{\gnumericColA}|}%
	{\gnumericPB{\raggedright}\gnumbox[l]{$R_{o}$}}
	&\multicolumn{1}{p{\gnumericColB}|}%
	{\gnumericPB{\raggedright}\gnumbox[l]{Output resistance of G circuit}}
\\
\hhline{|--|}
	 \multicolumn{1}{|p{\gnumericColA}|}%
	{\gnumericPB{\raggedright}\gnumbox[l]{$R_{if}$}}
	&\multicolumn{1}{p{\gnumericColB}|}%
	{\gnumericPB{\raggedright}\gnumbox[l]{Input resistance of Feedback}}
\\
\hhline{|--|}
	 \multicolumn{1}{|p{\gnumericColA}|}%
	{\gnumericPB{\raggedright}\gnumbox[l]{$R_{of}$}}
	&\multicolumn{1}{p{\gnumericColB}|}%
	{\gnumericPB{\raggedright}\gnumbox[l]{Output resistance of Feedback}}
\\
\hhline{|--|}
	 \multicolumn{1}{|p{\gnumericColA}|}%
	{\gnumericPB{\raggedright}\gnumbox[l]{$R_{s}$}}
	&\multicolumn{1}{p{\gnumericColB}|}%
	{\gnumericPB{\raggedright}\gnumbox[l]{Resistance of Current Source}}
\\
\hhline{|--|}
	 \multicolumn{1}{|p{\gnumericColA}|}%
	{\gnumericPB{\raggedright}\gnumbox[l]{$R_{L}$}}
	&\multicolumn{1}{p{\gnumericColB}|}%
	{\gnumericPB{\raggedright}\gnumbox[l]{Output Load Resistance}}
\\
\hhline{|--|}
	 \multicolumn{1}{|p{\gnumericColA}|}%
	{\gnumericPB{\raggedright}\gnumbox[l]{$g_{m}$}}
	&\multicolumn{1}{p{\gnumericColB}|}%
	{\gnumericPB{\raggedright}\gnumbox[l]{Trans conductance }}
\\
\hhline{|--|}
	 \multicolumn{1}{|p{\gnumericColA}|}%
	{\gnumericPB{\raggedright}\gnumbox[l]{$I_{C}$}}
	&\multicolumn{1}{p{\gnumericColB}|}%
	{\gnumericPB{\raggedright}\gnumbox[l]{Collector current }}
\\
\hhline{|--|}
	 \multicolumn{1}{|p{\gnumericColA}|}%
	{\gnumericPB{\raggedright}\gnumbox[l]{$I_{E}$}}
	&\multicolumn{1}{p{\gnumericColB}|}%
	{\gnumericPB{\raggedright}\gnumbox[l]{Emitter Current }}
\\
\hhline{|--|}
	 \multicolumn{1}{|p{\gnumericColA}|}%
	{\gnumericPB{\raggedright}\gnumbox[l]{$I_{B}$}}
	&\multicolumn{1}{p{\gnumericColB}|}%
	{\gnumericPB{\raggedright}\gnumbox[l]{Base Current }}
\\

\hhline{|-|-|}
\end{tabular}

\ifthenelse{\isundefined{\languageshorthands}}{}{\languageshorthands{\languagename}}
\gnumericTableEnd

\caption{}
\label{table:ee17btech11031_spice_Table}
\end{table}

\item Find the small-signal equivalent circuit of the amplifier with the signal source represented by its Norton equivalent (as we usually do when the feedback connection at the input is shunt).

\solution
In fig \ref{fig:sss ckt} 
 \begin{figure}[!ht]
 	\begin{center}
 			\resizebox{\columnwidth}{!}{\begin{circuitikz}
\ctikzset{tripoles/mos style/arrows}

\draw (0,0) node [ground]{};
\draw (0,0) -- (0,1) to[isource, l = $I_{S}$] (0,2) -- (0,3);
\draw (0,3) -- (1.5,3) to[R = $R_{S}$] (1.5,0) node [ground]{};
\draw (1.5,3) -- (3,3) to[R = $r_{\pi}$] (3,0) node [ground]{} ;
\draw node at (3.9,2){$+$};
\draw node at (3.9, 1.5) {$V_{\pi}$};
\draw node at (3.9, 1) {$-$};
\draw (6,3) -- (5,3) to[cisource , l= $g_{m}V_{\pi}$] (5,0) node [ground]{};
\draw (6,3) -- (6.5,3) to [R = $R_{C}$] (6.5,0) node [ground]{};
\draw (2,3) -- (2, 4) to [R = $R_{f}$] (5.5,4)--(5.5,3) ;
\draw (6,3)  to[short, -o] (7.5,3) ;
\draw node at (7.9,3) {$V_{o}$};
\draw (7.8,0)node[label = {right:$R_{of}$}]{} --(7.8,1.9);
\draw (7.5,2) ++(0,-0.1) node[flowarrow, rotate=180]{};
\draw (0.5,0)node[label = {right:$R_{if}$}]{} --(0.5,0.9);
\draw (0.85,0) ++(0,0.9) node[flowarrow]{};

\end{circuitikz}
}
 	\end{center}
 \caption{}
 \label{fig:sss ckt}
 \end{figure}

\item Find the G circuit and determine the value of G, $R_{i}$ ,
and $R_{o}$.

\solution
G circuit in fig. \ref{fig: G ckt} and G Block in \ref{fig: G Block}
 \begin{figure}[!ht]
 	\begin{center}
 			\resizebox{\columnwidth}{!}{\begin{circuitikz}[american]
\ctikzset{tripoles/mos style/arrows}

\draw (0,-2.5) -- (0,-1.5) to[isource, l = $I_{i}$] (0,-1) -- (0,0);
\draw (0,-2.5) -- (2, -2.5) -- (2, -2) to[R = $R_{s}$] (2, -1) -- (2, 0);
\draw (2, -2.5) -- (4, -2.5) -- (4, -2) to[R = $R_{11}$] (4, -1) -- (4, 0);
\draw (0, 0) -- (6,0);
\draw (4, -2.5) -- (6, -2.5);

\draw (10, 0) -- (12, 0) -- (12, -1) to[R = $R_{22}$] (12,-2) -- (12, -2.5);
\draw (12, 0) -- (14, 0) -- (14, -1) to[R = $R_{L}$] (14, -2) -- (14, -2.5);
\draw (10, -2.5) -- (14, -2.5);
\draw (8,-1)node[draw,minimum width=4cm,minimum height=4cm] (load) {Basic Amplifier}(8,0); 
\draw (14, -2.5) to[short, -o] (16,-2.5);
\draw (14, 0) to[short, -o] (16, 0);
\draw node at(16, -0.5){$+$};
\draw node at(16, -1.5){$V_{o}$};
\draw node at(16, -2.3){$-$};
\draw (0.8,-3.5)node[label = {right:$R_{i}$}]{} --(0.8,-2);
\draw (2.1,-2.1) ++(-1,0.1) node[flowarrow]{};
\draw (15.4,-3.5)node[label = {right:$R_{o}$}]{} --(15.4,-2);
\draw (16,-2.1) ++(-1,0.1) node[flowarrow, rotate = 180]{};

\end{circuitikz}}
 	\end{center}
 \caption{Gain Block}
 \label{fig: G Block}
 \end{figure}

 \begin{figure}[!ht]
 	\begin{center}
 			\resizebox{\columnwidth}{!}{\begin{circuitikz}
\ctikzset{tripoles/mos style/arrows}

\draw (0,0) node [ground]{};
\draw (0,3) to[short, -o] (-1,3);
\draw (0,0) -- (0,1) to[R = $R_{S}$] (0,2) -- (0,3);
\draw (0,3) -- (1.5,3) to[R = $R_{f}$] (1.5,0) node [ground]{};
\draw (1.5,3) -- (3,3) to[R = $r_{\pi}$] (3,0) node [ground]{} ;
\draw node at (3.9,2){$+$};
\draw node at (3.9, 1.5) {$V_{\pi}$};
\draw node at (3.9, 1) {$-$};
\draw node at (8.9, 3) {$V_{o}$};
\draw (6,3) -- (5,3) to[cisource , l= $g_{m}V_{\pi}$] (5,0) node [ground]{};
\draw (6,3) -- (6.5,3) to [R = $R_{f}$] (6.5,0) node [ground]{};
\draw (6.5, 3) -- (7.5, 3);
\draw node at (-1,4) {$I_{i}$};
\draw (7.5,3) -- (7.8,3) to[R = $R_{C}$] (7.8, 0) node [ground]{};
\draw (7.8, 3)  to[short, -o] (8.5, 3);
\draw [-latex] (-1.5, 3.5) -- (-0.5,3.5);
\draw (9.30,0)node[label = {right:$R_{o}$}]{} --(9.30,1.9);
\draw (9,2) ++(0,-0.1) node[flowarrow, rotate=180]{};
\draw (-1.3,0)node[label = {right:$R_{i}$}]{} --(-1.3,1.1);
\draw (0,1) ++(-1,0.1) node[flowarrow]{};
\end{circuitikz}
}
 	\end{center}
 \caption{G Circuit}
 \label{fig: G ckt}
 \end{figure}

\begin{align}
    g_{m} &= \frac{I_{C}}{V_{\pi}}
    = \frac{1.606*10^{-3}}{25*10^{-3}}
    = 64 mA/V\\
    r_{\pi} &= \frac{H    }{g_{m}} = \frac{100}{64*10^{-3}}
    = 1.56 k \ohm\\
     V_{o} &= -g_{m}V_{\pi}\brak{R_{f} || R_{C}} \\
    V_{\pi} &= I_{i}\brak{R_{S} || R_{f} || r_{\pi}}
\end{align}
\begin{align}
    \text{Gain}, G &= \frac{V_{o}}{I_{l}}\\
    G &= -g_{m} \brak{R_{f} || R_{c}} \brak{R_{s}|| R_{f} || r_{s}}\\
    G &= -429 k\ohm
\end{align}

Input Resistance
\begin{align}
    R_{i} = \brak{R_{s} || R_{f} || r_{s}} = 1.31 k\ohm\\
    \text{Output Resistance},\\
    R_{o} = R_{C} || R_{f}\\
    R_{o} = 5.09 k\ohm
\end{align}

\item Find H and hence AH and 1+AH.

\solution 
 \begin{figure}[!ht]
 	\begin{center}
 			\resizebox{\columnwidth}{!}{\begin{circuitikz}[american]
\ctikzset{tripoles/mos style/arrows}

\draw (0,0) to[short, i = $I_{f}$] (0,2) -- (2,2)-- (5,2) {}
(7,1)node[draw,minimum width=4cm,minimum height=4cm] (load) {Feedback Network}{}
(0,0)--(5,0){}
(14,0) to[V = $V_{o}$] (14,2) -- (9,2)
(14,0) -- (9,0){}
node at(2,1.8){$+$}
node at(2,0.2){$-$}
node at(2, 1){$V_{o}$}
node at(2.5, 1){$=$}
node at(3, 1){$0$}
;\end{circuitikz}}
 	\end{center}
 \caption{H Block}
 \label{fig:  H blk}
 \end{figure}
 
  \begin{figure}[!ht]
 	\begin{center}
 			\resizebox{\columnwidth}{!}{\begin{circuitikz}[american]
\ctikzset{tripoles/mos style/arrows}

\draw (0,0) node[ground]{};
\draw (0,0) to[short, i = $I_{f}$] (0,2.5){};
\draw (0, 2.5) to[R = $R_{f}$] (2.5, 2.5);
\draw (2.5,2.5) to[short, -o] (3,2.5);
\draw (3,0) node[ground]{};
\draw node at(3,2.2){$+$};
\draw node at(3,1.4){$V_{o}$};
\draw node at(3,0.5){$-$};

\end{circuitikz}}
 	\end{center}
 \caption{H Circuit}
 \label{fig:  H ckt}
 \end{figure}
 
\begin{align}
    H &= \frac{I_{f}}{V_{o}}
    = -\frac{1}{R_{f}}\\
    \implies H &= -17.85*10^{-4}\\
    GH &= 7.662\\
    1 + GH &= 8.66
\end{align}

\item Find $R_{11}$ and $R_{22}$ from fig. \ref{fig:  H ckt}

\solution 

\begin{align}
    R_{11} &= R_{f}\\
    R_{22} &= R_{f}
\end{align}

\item Find T , $R_{if}$  and $R_{of}$ and hence $R_{in}$ and $R_{out}$ .

\solution 
\begin{align}
    T &= \frac{G}{1+GH}\\
    &= -49.54 k\ohm\\
    R_{if} &= \frac{R_{i}}{1 + GH}\\
    &= \frac{1.31*10^3}{8.66}\\
    &= 151.27 \ohm\\
    R_{of} &= \frac{R_{o}}{1+GH}\\
    &= \frac{5.09*10^3}{8.66}\\
    &= 587.7 \ohm\\
    R_{in} &= \frac{1}{\frac{1}{R_{if}} - \frac{1}{R_{s}}}\\
    &=153.2 \ohm\\
    R_{out} &= \frac{1}{\frac{1}{R_{of}} - \frac{1}{R_{L}}}\\
    &= R_{of}
\end{align}

\item What voltage gain $V_{o}$/$V_{s}$ is realized? How does this value
compare to the ideal value obtained if the loop gain is
very large and thus the signal voltage at the base becomes
almost zero (like what happens in an inverting op-amp
circuit).

\solution 
\begin{align}
    \frac{V_{o}}{V_{s}} &= \frac{V_{o}}{I_{s}R_{s}}\\
    &= \frac{T}{R_{s}}\\
    \text{Since}, T &= \frac{V_{o}}{I_{S}}\\
    \frac{V_{o}}{V_{s}} &= \frac{-49.54*10^3}{10*10^3}\\
    &= -4.95 V/V
\end{align}

If the loop gain is very large, then the gain with feedback T is:

\begin{align}
    T &= \frac{1}{H}\\
    &= \frac{1}{\brak{-17.85*10^{-6}}}\\
    &= -56k\ohm
\end{align}
$\therefore$  the closed loop gain,  T = - $R_{f}$

\begin{table}[!ht]
\centering
%%%%%%%%%%%%%%%%%%%%%%%%%%%%%%%%%%%%%%%%%%%%%%%%%%%%%%%%%%%%%%%%%%%%%%
%%                                                                  %%
%%  This is the header of a LaTeX2e file exported from Gnumeric.    %%
%%                                                                  %%
%%  This file can be compiled as it stands or included in another   %%
%%  LaTeX document. The table is based on the longtable package so  %%
%%  the longtable options (headers, footers...) can be set in the   %%
%%  preamble section below (see PRAMBLE).                           %%
%%                                                                  %%
%%  To include the file in another, the following two lines must be %%
%%  in the including file:                                          %%
%%        \def\inputGnumericTable{}                                 %%
%%  at the beginning of the file and:                               %%
%%        \input{name-of-this-file.tex}                             %%
%%  where the table is to be placed. Note also that the including   %%
%%  file must use the following packages for the table to be        %%
%%  rendered correctly:                                             %%
%%    \usepackage[latin1]{inputenc}                                 %%
%%    \usepackage{color}                                            %%
%%    \usepackage{array}                                            %%
%%    \usepackage{longtable}                                        %%
%%    \usepackage{calc}                                             %%
%%    \usepackage{multirow}                                         %%
%%    \usepackage{hhline}                                           %%
%%    \usepackage{ifthen}                                           %%
%%  optionally (for landscape tables embedded in another document): %%
%%    \usepackage{lscape}                                           %%
%%                                                                  %%
%%%%%%%%%%%%%%%%%%%%%%%%%%%%%%%%%%%%%%%%%%%%%%%%%%%%%%%%%%%%%%%%%%%%%%

 

%%  This section checks if we are begin input into another file or  %%
%%  the file will be compiled alone. First use a macro taken from   %%
%%  the TeXbook ex 7.7 (suggestion of Han-Wen Nienhuys).            %%
\def\ifundefined#1{\expandafter\ifx\csname#1\endcsname\relax}


%%  Check for the \def token for inputed files. If it is not        %%
%%  defined, the file will be processed as a standalone and the     %%
%%  preamble will be used.                                          %%
\ifundefined{inputGnumericTable}

%%  We must be able to close or not the document at the end.        %%
	\def\gnumericTableEnd{\end{document}}


%%%%%%%%%%%%%%%%%%%%%%%%%%%%%%%%%%%%%%%%%%%%%%%%%%%%%%%%%%%%%%%%%%%%%%
%%                                                                  %%
%%  This is the PREAMBLE. Change these values to get the right      %%
%%  paper size and other niceties.                                  %%
%%                                                                  %%
%%%%%%%%%%%%%%%%%%%%%%%%%%%%%%%%%%%%%%%%%%%%%%%%%%%%%%%%%%%%%%%%%%%%%%

	\documentclass[12pt%
			  %,landscape%
                    ]{report}
       \usepackage[latin1]{inputenc}
       \usepackage{fullpage}
       \usepackage{color}
       \usepackage{array}
       \usepackage{longtable}
       \usepackage{calc}
       \usepackage{multirow}
       \usepackage{hhline}
       \usepackage{ifthen}

	


%%  End of the preamble for the standalone. The next section is for %%
%%  documents which are included into other LaTeX2e files.          %%
\else

%%  We are not a stand alone document. For a regular table, we will %%
%%  have no preamble and only define the closing to mean nothing.   %%
    \def\gnumericTableEnd{}

%%  If we want landscape mode in an embedded document, comment out  %%
%%  the line above and uncomment the two below. The table will      %%
%%  begin on a new page and run in landscape mode.                  %%
%       \def\gnumericTableEnd{\end{landscape}}
%       \begin{landscape}


%%  End of the else clause for this file being \input.              %%
\fi

%%%%%%%%%%%%%%%%%%%%%%%%%%%%%%%%%%%%%%%%%%%%%%%%%%%%%%%%%%%%%%%%%%%%%%
%%                                                                  %%
%%  The rest is the gnumeric table, except for the closing          %%
%%  statement. Changes below will alter the table's appearance.     %%
%%                                                                  %%
%%%%%%%%%%%%%%%%%%%%%%%%%%%%%%%%%%%%%%%%%%%%%%%%%%%%%%%%%%%%%%%%%%%%%%

\providecommand{\gnumericmathit}[1]{#1} 
%%  Uncomment the next line if you would like your numbers to be in %%
%%  italics if they are italizised in the gnumeric table.           %%
%\renewcommand{\gnumericmathit}[1]{\mathit{#1}}
\providecommand{\gnumericPB}[1]%
{\let\gnumericTemp=\\#1\let\\=\gnumericTemp\hspace{0pt}}
 \ifundefined{gnumericTableWidthDefined}
        \newlength{\gnumericTableWidth}
        \newlength{\gnumericTableWidthComplete}
        \newlength{\gnumericMultiRowLength}
        \global\def\gnumericTableWidthDefined{}
 \fi
%% The following setting protects this code from babel shorthands.  %%
 \ifthenelse{\isundefined{\languageshorthands}}{}{\languageshorthands{english}}
%%  The default table format retains the relative column widths of  %%
%%  gnumeric. They can easily be changed to c, r or l. In that case %%
%%  you may want to comment out the next line and uncomment the one %%
%%  thereafter                                                      %%
\providecommand\gnumbox{\makebox[0pt]}
%%\providecommand\gnumbox[1][]{\makebox}

%% to adjust positions in multirow situations                       %%
\setlength{\bigstrutjot}{\jot}
\setlength{\extrarowheight}{\doublerulesep}

%%  The \setlongtables command keeps column widths the same across  %%
%%  pages. Simply comment out next line for varying column widths.  %%
\setlongtables

\setlength\gnumericTableWidth{%
	83pt+%
	91pt+%
0pt}
\def\gumericNumCols{2}
\setlength\gnumericTableWidthComplete{\gnumericTableWidth+%
         \tabcolsep*\gumericNumCols*2+\arrayrulewidth*\gumericNumCols}
\ifthenelse{\lengthtest{\gnumericTableWidthComplete > \linewidth}}%
         {\def\gnumericScale{\ratio{\linewidth-%
                        \tabcolsep*\gumericNumCols*2-%
                        \arrayrulewidth*\gumericNumCols}%
{\gnumericTableWidth}}}%
{\def\gnumericScale{1}}

%%%%%%%%%%%%%%%%%%%%%%%%%%%%%%%%%%%%%%%%%%%%%%%%%%%%%%%%%%%%%%%%%%%%%%
%%                                                                  %%
%% The following are the widths of the various columns. We are      %%
%% defining them here because then they are easier to change.       %%
%% Depending on the cell formats we may use them more than once.    %%
%%                                                                  %%
%%%%%%%%%%%%%%%%%%%%%%%%%%%%%%%%%%%%%%%%%%%%%%%%%%%%%%%%%%%%%%%%%%%%%%

\ifthenelse{\isundefined{\gnumericColA}}{\newlength{\gnumericColA}}{}\settowidth{\gnumericColA}{\begin{tabular}{@{}p{83pt*\gnumericScale}@{}}x\end{tabular}}
\ifthenelse{\isundefined{\gnumericColB}}{\newlength{\gnumericColB}}{}\settowidth{\gnumericColB}{\begin{tabular}{@{}p{91pt*\gnumericScale}@{}}x\end{tabular}}

\begin{tabular}[c]{%
	b{\gnumericColA}%
	b{\gnumericColB}%
	}

%%%%%%%%%%%%%%%%%%%%%%%%%%%%%%%%%%%%%%%%%%%%%%%%%%%%%%%%%%%%%%%%%%%%%%
%%  The longtable options. (Caption, headers... see Goosens, p.124) %%
%	\caption{The Table Caption.}             \\	%
% \hline	% Across the top of the table.
%%  The rest of these options are table rows which are placed on    %%
%%  the first, last or every page. Use \multicolumn if you want.    %%

%%  Header for the first page.                                      %%
%	\multicolumn{2}{c}{The First Header} \\ \hline 
%	\multicolumn{1}{c}{colTag}	%Column 1
%	&\multicolumn{1}{c}{colTag}	\\ \hline %Last column
%	\endfirsthead

%%  The running header definition.                                  %%
%	\hline
%	\multicolumn{2}{l}{\ldots\small\slshape continued} \\ \hline
%	\multicolumn{1}{c}{colTag}	%Column 1
%	&\multicolumn{1}{c}{colTag}	\\ \hline %Last column
%	\endhead

%%  The running footer definition.                                  %%
%	\hline
%	\multicolumn{2}{r}{\small\slshape continued\ldots} \\
%	\endfoot

%%  The ending footer definition.                                   %%
%	\multicolumn{2}{c}{That's all folks} \\ \hline 
%	\endlastfoot
%%%%%%%%%%%%%%%%%%%%%%%%%%%%%%%%%%%%%%%%%%%%%%%%%%%%%%%%%%%%%%%%%%%%%%

\hhline{|-|-}
	 \multicolumn{1}{|p{\gnumericColA}|}%
	{\gnumericPB{\raggedright}\gnumbox[l]{\textbf{Parameter}}}
	&\multicolumn{1}{p{\gnumericColA}|}%
	{\gnumericPB{\raggedright}\gnumbox[l]{\textbf{Value}}}
\\
\hhline{|-|-}
	 \multicolumn{1}{|p{\gnumericColA}|}%
	{\gnumericPB{\raggedright}\gnumbox[l]{\textbf{$R_{f}$}}}
	&\multicolumn{1}{p{\gnumericColA}|}%
	{\gnumericPB{\raggedright}\gnumbox[l]{\textbf{$56k\Omega$}}}
\\
\hhline{|-|-}
	 \multicolumn{1}{|p{\gnumericColA}|}%
	{\gnumericPB{\raggedright}\gnumbox[l]{\textbf{$R_{S}$}}}
	&\multicolumn{1}{p{\gnumericColA}|}%
	{\gnumericPB{\raggedright}\gnumbox[l]{\textbf{$10k\Omega$}}}
\\
\hhline{|-|-}
	 \multicolumn{1}{|p{\gnumericColA}|}%
	{\gnumericPB{\raggedright}\gnumbox[l]{\textbf{$R_{C}$}}}
	&\multicolumn{1}{p{\gnumericColA}|}%
	{\gnumericPB{\raggedright}\gnumbox[l]{\textbf{$5.6k\Omega$}}}
\\
\hhline{|-|-}
	 \multicolumn{1}{|p{\gnumericColA}|}%
	{\gnumericPB{\raggedright}\gnumbox[l]{\textbf{$I_{S}$}}}
	&\multicolumn{1}{p{\gnumericColA}|}%
	{\gnumericPB{\raggedright}\gnumbox[l]{\textbf{$0.07mA$}}}
\\
\hhline{|-|-}
	 \multicolumn{1}{|p{\gnumericColA}|}%
	{\gnumericPB{\raggedright}\gnumbox[l]{\textbf{$I_{B}$}}}
	&\multicolumn{1}{p{\gnumericColB}|}%
	{\gnumericPB{\raggedright}\gnumbox[l]{\textbf{$16.06\mu A$}}}
\\
\hhline{|-|-}
	 \multicolumn{1}{|p{\gnumericColA}|}%
	{\gnumericPB{\raggedright}\gnumbox[l]{\textbf{$I_{C}$}}}
	&\multicolumn{1}{p{\gnumericColA}|}%
	{\gnumericPB{\raggedright}\gnumbox[l]{\textbf{$1.606 mA$}}}
\\
\hhline{|-|-}
	 \multicolumn{1}{|p{\gnumericColA}|}%
	{\gnumericPB{\raggedright}\gnumbox[l]{\textbf{$I_{E}$}}}
	&\multicolumn{1}{p{\gnumericColA}|}%
	{\gnumericPB{\raggedright}\gnumbox[l]{\textbf{$1.606 mA$}}}
\\
\hhline{|-|-}
	 \multicolumn{1}{|p{\gnumericColA}|}%
	{\gnumericPB{\raggedright}\gnumbox[l]{\textbf{$g_{m}$}}}
	&\multicolumn{1}{p{\gnumericColA}|}%
	{\gnumericPB{\raggedright}\gnumbox[l]{\textbf{$64 mA/V$ }}}
\\

\hhline{|-|-}
	 \multicolumn{1}{|p{\gnumericColA}|}%
	{\gnumericPB{\raggedright}\gnumbox[l]{\textbf{$r_{\pi}$}}}
	&\multicolumn{1}{p{\gnumericColA}|}%
	{\gnumericPB{\raggedright}\gnumbox[l]{\textbf{$1.56k\Omega$}}}
\\
\hhline{|-|-}
	 \multicolumn{1}{|p{\gnumericColA}|}%
	{\gnumericPB{\raggedright}\gnumbox[l]{\textbf{$G$}}}
	&\multicolumn{1}{p{\gnumericColA}|}%
	{\gnumericPB{\raggedright}\gnumbox[l]{\textbf{$-429k\Omega$}}}
\\
\hhline{|-|-}
	 \multicolumn{1}{|p{\gnumericColA}|}%
	{\gnumericPB{\raggedright}\gnumbox[l]{\textbf{$R_{i}$}}}
	&\multicolumn{1}{p{\gnumericColA}|}%
	{\gnumericPB{\raggedright}\gnumbox[l]{\textbf{$1.31k\Omega$}}}
\\
\hhline{|-|-}
	 \multicolumn{1}{|p{\gnumericColA}|}%
	{\gnumericPB{\raggedright}\gnumbox[l]{\textbf{$R_{o}$}}}
	&\multicolumn{1}{p{\gnumericColA}|}%
	{\gnumericPB{\raggedright}\gnumbox[l]{\textbf{$5.09k\Omega$}}}
\\

\hhline{|-|-}
	 \multicolumn{1}{|p{\gnumericColA}|}%
	{\gnumericPB{\raggedright}\gnumbox[l]{\textbf{$H$}}}
	&\multicolumn{1}{p{\gnumericColA}|}%
	{\gnumericPB{\raggedright}\gnumbox[l]{\textbf{$-17.85*10^{-4}$}}}
\\
\hhline{|-|-}
	 \multicolumn{1}{|p{\gnumericColA}|}%
	{\gnumericPB{\raggedright}\gnumbox[l]{\textbf{$GH$}}}
	&\multicolumn{1}{p{\gnumericColA}|}%
	{\gnumericPB{\raggedright}\gnumbox[l]{\textbf{$7.662$}}}
\\
\hhline{|-|-}
	 \multicolumn{1}{|p{\gnumericColA}|}%
	{\gnumericPB{\raggedright}\gnumbox[l]{\textbf{$1+GH$}}}
	&\multicolumn{1}{p{\gnumericColA}|}%
	{\gnumericPB{\raggedright}\gnumbox[l]{\textbf{$8.66$}}}
\\

\hhline{|-|-}
	 \multicolumn{1}{|p{\gnumericColA}|}%
	{\gnumericPB{\raggedright}\gnumbox[l]{\textbf{$T$}}}
	&\multicolumn{1}{p{\gnumericColA}|}%
	{\gnumericPB{\raggedright}\gnumbox[l]{\textbf{$-49.54k\Omega$}}}
\\
\hhline{|-|-}
	 \multicolumn{1}{|p{\gnumericColA}|}%
	{\gnumericPB{\raggedright}\gnumbox[l]{\textbf{$R_{if}$}}}
	&\multicolumn{1}{p{\gnumericColA}|}%
	{\gnumericPB{\raggedright}\gnumbox[l]{\textbf{$151.27k\Omega$}}}
\\
\hhline{|-|-}
	 \multicolumn{1}{|p{\gnumericColA}|}%
	{\gnumericPB{\raggedright}\gnumbox[l]{\textbf{$R_{of}$}}}
	&\multicolumn{1}{p{\gnumericColA}|}%
	{\gnumericPB{\raggedright}\gnumbox[l]{\textbf{$587.7k\Omega$}}}
\\
\hhline{|-|-}
	 \multicolumn{1}{|p{\gnumericColA}|}%
	{\gnumericPB{\raggedright}\gnumbox[l]{\textbf{$R_{in}$}}}
	&\multicolumn{1}{p{\gnumericColA}|}%
	{\gnumericPB{\raggedright}\gnumbox[l]{\textbf{$153.2 \Omega$}}}
\\

\hhline{|-|-|}
\end{tabular}

\ifthenelse{\isundefined{\languageshorthands}}{}{\languageshorthands{\languagename}}
\gnumericTableEnd
\caption{}
\label{table:ee17btech11031_value_Table}
\end{table}

 \begin{figure}[!ht]
 	\begin{center}
 			\resizebox{\columnwidth}{!}{\begin{circuitikz}

\draw
(0,0) to[I = $I_{s}$] (0,2) -- (2,2)
(2,2) -- (5,2) {}
(7,1)node[draw,minimum width=4cm,minimum height=4cm] (load) {Gain Amplifier}{}
(7,-4)node[draw,minimum width=4cm,minimum height=4cm] (load) {Feedback Network}{}
(0,0)--(5,0)
(14,0) to[R=$R_L$,*-*] (14,2) -- (9,2)
(3,0) -- (3,-5) -- (5,-5){}
(4,0)--(4,-3) -- (5,-3){}
(2,0) to[R = $R_{S}$] (2,1.8) -- (2,2)
(9,-3) -- (10,-3) -- (10,0){}
(9,0) -- (10,0) -- (10,2){}
(10, 0) -- (11,0){}
(9,-5) -- (11,-5) -- (11,0) -- (14,0){}
(4,-2) to[short, i=$I_{f}$] (4,-1)
(4,0) -- (4,2)
node at(9.5, -3.2){$+$}
node at(9.5, -4.1){$V_{o}$}
node at(9.5, -4.9){$-$}
node at(12.5, 0.2){$-$}
node at(12.5, 1){$V_{o}$}
node at(12.5, 1.5){$+$}
;
\end{circuitikz}
}
 	\end{center}
 \caption{Shunt-Shunt Amplifier Block Diagram}
 \label{fig: ss2 ckt}
 \end{figure}

 \begin{figure}[!ht]
 	\begin{center}
 			\resizebox{\columnwidth}{!}{  
\begin{circuitikz}[american]
\usetikzlibrary{positioning, fit, calc}
\draw (0,-2) to[isource, l = $I{s}$] (0,0);
\draw (0,0) -- (6,0);
\draw (6,0) to[R = $R_{i}$] (6,-2) -- (0,-2);
\draw (6, -4) to[controlled current source = $HV_{o}$] (6,-6);
\draw (6,-4) -- (4,-4) -- (4, 0);
\draw (6,-6) -- (2, -6 ) -- (2, -2);
\draw node at(5,-0.2){$+$};
\draw node at(5,-1.8){$-$};
\draw node at(5,-1){$V_{i}$};
\draw (10,0) to [controlled voltage source=$GV_i$]++(0,-2);
\draw (12, 0) to[R = $R_{o}$] (14,0);
\draw (10,0) -- (12,0);
\draw (14,0)--(14,-2)--(14,-4)--(12,-4);
\draw (12, -6) -- (16,-6) -- (16, -2);

\draw (12, -4) to[short, -o] (12, -4);
\draw (12, -6) to[short, -o] (12, -6);
\draw (14,-2) -- (16,-2);
\draw (16, -2) to[short, -o] (18,-2);
\draw (16, 0) to[short, -o] (18, 0);
\draw node at(18,-0.2){$+$};
\draw (14,0) -- (16, 0);
\draw node at(18, -1.8){$-$};
\draw node at(18, - 1){$V_{o}$};
\draw (10,-2)--(12,-2);
\draw (12,-2) -- (14,-2);
\draw (5,0)coordinate(left);
\draw (5,-2)coordinate(bottoml);
\draw (13,0)coordinate(right);
\draw (13,-2)coordinate(bottomr);
\draw node[fit=(left)(right)(bottoml)(bottomr),draw, dashed, label={amplifier circuit},inner sep=10pt] {};
\draw (5,-4)coordinate(left1);
\draw (5,-6)coordinate(bottoml1);
\draw (13,-4)coordinate(right1);
\draw (13,-6)coordinate(bottomr1);
\draw node[fit=(left1)(right1)(bottoml1)(bottomr1),draw, dashed, label={feedback circuit},inner sep=10pt] {};


\end{circuitikz}}
 	\end{center}
 \caption{Ideal structure for Shunt-Shunt}
 \label{fig:ideal ckt}
 \end{figure}
 
\item Verify your solution using spice

\solution Doing operational Analysis on the Circuit \ref{fig:Original ckt}

\begin{table}[!ht]
\centering
%%%%%%%%%%%%%%%%%%%%%%%%%%%%%%%%%%%%%%%%%%%%%%%%%%%%%%%%%%%%%%%%%%%%%%
%%                                                                  %%
%%  This is the header of a LaTeX2e file exported from Gnumeric.    %%
%%                                                                  %%
%%  This file can be compiled as it stands or included in another   %%
%%  LaTeX document. The table is based on the longtable package so  %%
%%  the longtable options (headers, footers...) can be set in the   %%
%%  preamble section below (see PRAMBLE).                           %%
%%                                                                  %%
%%  To include the file in another, the following two lines must be %%
%%  in the including file:                                          %%
%%        \def\inputGnumericTable{}                                 %%
%%  at the beginning of the file and:                               %%
%%        \input{name-of-this-file.tex}                             %%
%%  where the table is to be placed. Note also that the including   %%
%%  file must use the following packages for the table to be        %%
%%  rendered correctly:                                             %%
%%    \usepackage[latin1]{inputenc}                                 %%
%%    \usepackage{color}                                            %%
%%    \usepackage{array}                                            %%
%%    \usepackage{longtable}                                        %%
%%    \usepackage{calc}                                             %%
%%    \usepackage{multirow}                                         %%
%%    \usepackage{hhline}                                           %%
%%    \usepackage{ifthen}                                           %%
%%  optionally (for landscape tables embedded in another document): %%
%%    \usepackage{lscape}                                           %%
%%                                                                  %%
%%%%%%%%%%%%%%%%%%%%%%%%%%%%%%%%%%%%%%%%%%%%%%%%%%%%%%%%%%%%%%%%%%%%%%

 

%%  This section checks if we are begin input into another file or  %%
%%  the file will be compiled alone. First use a macro taken from   %%
%%  the TeXbook ex 7.7 (suggestion of Han-Wen Nienhuys).            %%
\def\ifundefined#1{\expandafter\ifx\csname#1\endcsname\relax}


%%  Check for the \def token for inputed files. If it is not        %%
%%  defined, the file will be processed as a standalone and the     %%
%%  preamble will be used.                                          %%
\ifundefined{inputGnumericTable}

%%  We must be able to close or not the document at the end.        %%
	\def\gnumericTableEnd{\end{document}}


%%%%%%%%%%%%%%%%%%%%%%%%%%%%%%%%%%%%%%%%%%%%%%%%%%%%%%%%%%%%%%%%%%%%%%
%%                                                                  %%
%%  This is the PREAMBLE. Change these values to get the right      %%
%%  paper size and other niceties.                                  %%
%%                                                                  %%
%%%%%%%%%%%%%%%%%%%%%%%%%%%%%%%%%%%%%%%%%%%%%%%%%%%%%%%%%%%%%%%%%%%%%%

	\documentclass[12pt%
			  %,landscape%
                    ]{report}
       \usepackage[latin1]{inputenc}
       \usepackage{fullpage}
       \usepackage{color}
       \usepackage{array}
       \usepackage{longtable}
       \usepackage{calc}
       \usepackage{multirow}
       \usepackage{hhline}
       \usepackage{ifthen}

	


%%  End of the preamble for the standalone. The next section is for %%
%%  documents which are included into other LaTeX2e files.          %%
\else

%%  We are not a stand alone document. For a regular table, we will %%
%%  have no preamble and only define the closing to mean nothing.   %%
    \def\gnumericTableEnd{}

%%  If we want landscape mode in an embedded document, comment out  %%
%%  the line above and uncomment the two below. The table will      %%
%%  begin on a new page and run in landscape mode.                  %%
%       \def\gnumericTableEnd{\end{landscape}}
%       \begin{landscape}


%%  End of the else clause for this file being \input.              %%
\fi

%%%%%%%%%%%%%%%%%%%%%%%%%%%%%%%%%%%%%%%%%%%%%%%%%%%%%%%%%%%%%%%%%%%%%%
%%                                                                  %%
%%  The rest is the gnumeric table, except for the closing          %%
%%  statement. Changes below will alter the table's appearance.     %%
%%                                                                  %%
%%%%%%%%%%%%%%%%%%%%%%%%%%%%%%%%%%%%%%%%%%%%%%%%%%%%%%%%%%%%%%%%%%%%%%

\providecommand{\gnumericmathit}[1]{#1} 
%%  Uncomment the next line if you would like your numbers to be in %%
%%  italics if they are italizised in the gnumeric table.           %%
%\renewcommand{\gnumericmathit}[1]{\mathit{#1}}
\providecommand{\gnumericPB}[1]%
{\let\gnumericTemp=\\#1\let\\=\gnumericTemp\hspace{0pt}}
 \ifundefined{gnumericTableWidthDefined}
        \newlength{\gnumericTableWidth}
        \newlength{\gnumericTableWidthComplete}
        \newlength{\gnumericMultiRowLength}
        \global\def\gnumericTableWidthDefined{}
 \fi
%% The following setting protects this code from babel shorthands.  %%
 \ifthenelse{\isundefined{\languageshorthands}}{}{\languageshorthands{english}}
%%  The default table format retains the relative column widths of  %%
%%  gnumeric. They can easily be changed to c, r or l. In that case %%
%%  you may want to comment out the next line and uncomment the one %%
%%  thereafter                                                      %%
\providecommand\gnumbox{\makebox[0pt]}
%%\providecommand\gnumbox[1][]{\makebox}

%% to adjust positions in multirow situations                       %%
\setlength{\bigstrutjot}{\jot}
\setlength{\extrarowheight}{\doublerulesep}

%%  The \setlongtables command keeps column widths the same across  %%
%%  pages. Simply comment out next line for varying column widths.  %%
\setlongtables

\setlength\gnumericTableWidth{%
	83pt+%
	91pt+%
0pt}
\def\gumericNumCols{2}
\setlength\gnumericTableWidthComplete{\gnumericTableWidth+%
         \tabcolsep*\gumericNumCols*2+\arrayrulewidth*\gumericNumCols}
\ifthenelse{\lengthtest{\gnumericTableWidthComplete > \linewidth}}%
         {\def\gnumericScale{\ratio{\linewidth-%
                        \tabcolsep*\gumericNumCols*2-%
                        \arrayrulewidth*\gumericNumCols}%
{\gnumericTableWidth}}}%
{\def\gnumericScale{1}}

%%%%%%%%%%%%%%%%%%%%%%%%%%%%%%%%%%%%%%%%%%%%%%%%%%%%%%%%%%%%%%%%%%%%%%
%%                                                                  %%
%% The following are the widths of the various columns. We are      %%
%% defining them here because then they are easier to change.       %%
%% Depending on the cell formats we may use them more than once.    %%
%%                                                                  %%
%%%%%%%%%%%%%%%%%%%%%%%%%%%%%%%%%%%%%%%%%%%%%%%%%%%%%%%%%%%%%%%%%%%%%%

\ifthenelse{\isundefined{\gnumericColA}}{\newlength{\gnumericColA}}{}\settowidth{\gnumericColA}{\begin{tabular}{@{}p{83pt*\gnumericScale}@{}}x\end{tabular}}
\ifthenelse{\isundefined{\gnumericColB}}{\newlength{\gnumericColB}}{}\settowidth{\gnumericColB}{\begin{tabular}{@{}p{91pt*\gnumericScale}@{}}x\end{tabular}}

\begin{tabular}[c]{%
	b{\gnumericColA}%
	b{\gnumericColB}%
	}

%%%%%%%%%%%%%%%%%%%%%%%%%%%%%%%%%%%%%%%%%%%%%%%%%%%%%%%%%%%%%%%%%%%%%%
%%  The longtable options. (Caption, headers... see Goosens, p.124) %%
%	\caption{The Table Caption.}             \\	%
% \hline	% Across the top of the table.
%%  The rest of these options are table rows which are placed on    %%
%%  the first, last or every page. Use \multicolumn if you want.    %%

%%  Header for the first page.                                      %%
%	\multicolumn{2}{c}{The First Header} \\ \hline 
%	\multicolumn{1}{c}{colTag}	%Column 1
%	&\multicolumn{1}{c}{colTag}	\\ \hline %Last column
%	\endfirsthead

%%  The running header definition.                                  %%
%	\hline
%	\multicolumn{2}{l}{\ldots\small\slshape continued} \\ \hline
%	\multicolumn{1}{c}{colTag}	%Column 1
%	&\multicolumn{1}{c}{colTag}	\\ \hline %Last column
%	\endhead

%%  The running footer definition.                                  %%
%	\hline
%	\multicolumn{2}{r}{\small\slshape continued\ldots} \\
%	\endfoot

%%  The ending footer definition.                                   %%
%	\multicolumn{2}{c}{That's all folks} \\ \hline 
%	\endlastfoot
%%%%%%%%%%%%%%%%%%%%%%%%%%%%%%%%%%%%%%%%%%%%%%%%%%%%%%%%%%%%%%%%%%%%%%

\hhline{|-|-}
	 \multicolumn{1}{|p{\gnumericColA}|}%
	{\gnumericPB{\raggedright}\gnumbox[l]{\textbf{Parameter}}}
	&\multicolumn{1}{p{\gnumericColA}|}%
	{\gnumericPB{\raggedright}\gnumbox[l]{\textbf{Value}}}
\\
\hhline{|-|-}
	 \multicolumn{1}{|p{\gnumericColA}|}%
	{\gnumericPB{\raggedright}\gnumbox[l]{\textbf{$I_{C}$}}}
	&\multicolumn{1}{p{\gnumericColA}|}%
	{\gnumericPB{\raggedright}\gnumbox[l]{\textbf{$2.1 mA$}}}
\\
\hhline{|-|-}
	 \multicolumn{1}{|p{\gnumericColA}|}%
	{\gnumericPB{\raggedright}\gnumbox[l]{\textbf{$I_{E}$}}}
	&\multicolumn{1}{p{\gnumericColA}|}%
	{\gnumericPB{\raggedright}\gnumbox[l]{\textbf{$2.1 mA$}}}
\\
\hhline{|-|-}
	 \multicolumn{1}{|p{\gnumericColA}|}%
	{\gnumericPB{\raggedright}\gnumbox[l]{\textbf{$I_{B}$}}}
	&\multicolumn{1}{p{\gnumericColA}|}%
	{\gnumericPB{\raggedright}\gnumbox[l]{\textbf{$2.1 \mu A$}}}
\\
\hhline{|-|-}
	 \multicolumn{1}{|p{\gnumericColA}|}%
	{\gnumericPB{\raggedright}\gnumbox[l]{\textbf{$I_{S}$}}}
	&\multicolumn{1}{p{\gnumericColA}|}%
	{\gnumericPB{\raggedright}\gnumbox[l]{\textbf{$0.07mA$}}}
\\

\hhline{|-|-|}
\end{tabular}

\ifthenelse{\isundefined{\languageshorthands}}{}{\languageshorthands{\languagename}}
\gnumericTableEnd
\caption{}
\label{table:ee17btech11031_spice_Table}
\end{table}

Table \ref{table:ee17btech11031_spice_Table} is close to the numerical Calculation done above.
 \end{enumerate}
